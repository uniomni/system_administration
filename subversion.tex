\chapter{Revision Control}

Revision control is essential to 
\begin{itemize} 
  \item Keep an audit trail of changes to information, i.e. what changes were made, who made them, when were they made and why they were made.
  \item Be able to roll-back unwanted changesets
  \item Allow simultaneous editing of information at different location
\end{itemize}

The revision control system used at AIFDR is Subversion \cite{} which is one of the most common systems in use.

Links to Subversion tutorials used:
\begin{itemize} 
  \item \url{http://www.howtogeek.com/howto/ubuntu/install-subversion-with-web-access-on-ubuntu}
  \item \url{http://odyniec.net/articles/ubuntu-subversion-server/}
  \item \url{}
\end{itemize} 

\section{Setting up the Subversion server}  

\section{Creating a new repository}

AIFDR holds all the Subversion repositories on the Web Server
under \texttt{/home/svn}.
The following steps are needed to create a new repository.
For the purpose of this example the repository is named abcd.

\begin{itemize} 
  \item \texttt{svnadmin create /home/svn/abcd}
  \item Add the following to the file 
    \texttt{/etc/apache2/mods-enabled/dav\_svn.conf}
\begin{verbatim} 
<Location /svn/abcd>
  DAV svn
  SVNPath /home/svn/abcd
  AuthType Basic
  AuthName "Subversion Repository"
  AuthUserFile /etc/apache2/dav_svn.passwd

  AuthzSVNAccessFile /etc/apache2/dav_svn.authz
  Require valid-user
</Location>
\end{verbatim} 
  \item Add the following to the file \texttt{/etc/apache2/dav\_svn.authz}
  \begin{verbatim}
[abcd:/]
@aifdr = rw
@guests = r
  \end{verbatim} 
  \item Set the ownership (and group) to www-data:
  \begin{verbatim}
    chown -R www-data:www-data /home/svn/abcd
  \end{verbatim}  
  \item Set permissions
  \begin{verbatim}
    chmod 770 /home/svn/abcd
  \end{verbatim}    
  \item Restart the web server using the command \texttt{/etc/init.d/apache2 restart}
  
\end{itemize} 

\section{Creating a new Subversion user}

A new Subversion user with new password can be created using:
\begin{verbatim} 
htpasswd -cm /etc/apache2/dav\_svn.passwd <username>
\end{verbatim} 
Note that you should only use the -c option the FIRST time that you create a user. After that you will only want to use the -m option to add users and 
passwords to the list.

Alternatively, an encrypted password can be created on 
a Ubuntu (or any other Debian derived) Linux machine using the command
\begin{verbatim}
  mkpasswd
\end{verbatim}
The desired username and the generated encrypted password can then be added
to the file \texttt{/etc/apache2/dav\_svn.passwd} as
\begin{verbatim} 
<user>:<encrypted password>
\end{verbatim} 

Update \texttt{/etc/apache2/dav\_svn.authz} by adding new user to
the group \texttt{aifdr}. 

\section{Using Subversion}
  
Interaction with the Subversion repositories can take place either 
through the command line client svn or through graphical 
user interfaces such as Tortoise (see Section \ref{sec:tortoise}). 
The repository can also be
read through TRAC (see Section \ref{TRAC}) or 
directly by a web browser using URLs of the form
\begin{verbatim} 
http://www.aifdr.org/svn/<repository name>
\end{verbatim}   

